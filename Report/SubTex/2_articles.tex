\subsection{Mô tả các điều khoản cho ngữ cảnh smart contract}

Trong ngữ cảnh của bài toán trên, ta có 2 đối tượng cần xem xét là bên thuê A và bên vận chuyển B. Giữa 2 đối tượng có những quan hệ và ràng buộc, cần được chuyển thành những điều khoản cụ thể được quy định trong smart contract. Smart contract đóng vai trò là bên thứ 3 quy định các điều khoản này, tạo nên niềm tin giữa 2 bên khi 2 bên không có niềm tin tưởng cho nhau. Cụ thể ta có thể cụ thể thành các điều khoản sau cho smart contract:


\subsubsection*{Article 1.}
Bên A đưa ra thông tin về hàng hóa cần bên B vận chuyển (gồm thông tin khách hàng, tính chất, số lượng, giá trị hàng hóa, thời gian dự kiến, địa điểm giao/nhận hàng). Tất cả các thông tin được smart contract ghi lại và public cho 2 phía.

\subsubsection*{Article 2.}
Bên B sau khi nhận thông tin hàng hóa được gửi từ bên A, bên B sẽ tiến hành xác nhận đơn hàng sẽ gửi lại thông tin đơn hàng kèm theo các thông tin bổ sung về thời gian chi tiết, chi phí. Tất cả thông tin được smart contract ghi nhận lại và public qua cho 2 phía.

\subsubsection*{Article 3.}
Bên A gửi toàn bộ số tiền vào smart contract mà bên B sẽ được nhận nếu bên B hoàn thành xong việc đúng như quy định trong hợp đồng.

\subsubsection*{Article 4.}
Bên B gửi vào smart contract số tiền đặt cọc tối thiểu phải bằng số tiền bồi thường tối đa mà bên B phải chịu nếu có phát sinh những vi phạm các điều khoản trong hợp đồng.

\subsubsection*{Article 5.}
Nếu đúng thời gian gửi hàng mà đúng 1 bên vẫn chưa gửi số tiền quy định tại điều 3 và 4 thì smart contract sẽ gửi tiền trả về cho bên kia. Hoặc nếu chưa có bên nào gửi tiền thì hợp đồng kết thúc.

\subsubsection*{Article 6.}
Nếu bên B đến nơi nhận hàng trễ hơn thời gian quy định thì smart contract tự động gửi số tiền ban đầu về cho cả hai bên và kết thúc hợp đồng.

\subsubsection*{Article 7.}
Nếu bên B đến địa điểm gửi hàng đúng giờ mà bên A vẫn chưa chuẩn bị hàng đầy đủ hoặc hàng không đúng như mô tả thì smart contract sẽ tự động chuyển 1 số tiền bồi thường chi phí di chuyển cho bên B. Sau đó gửi số tiền còn lại cho mỗi bên rồi kết thúc hợp đồng.

\subsubsection*{Article 8.}
Nếu bên B giao hàng trễ hơn so với thời gian quy định nhưng không quá thời gian cho phép, hoặc không có hàng để giao hoặc làm thất thoát số hàng thì bên B phải bồi thường thiệt hại cho bên A với khoản tiền tương ứng. Trường hợp nếu bên B không giao hàng, thì phần chi phí vận chuyển sẽ được trả về cho bên A. 

\subsubsection*{Article 9.}
Sau khi bên B hoàn thành xong quá trình giao hàng thì smart contract sẽ gửi toàn bộ số tiền qua cho bên B. (gồm tiền chuyển hàng của A và tiền đặt cọc ban đầu của B trừ đi các khoản bồi thường nếu có). Bên A không nhận thêm bất cứ khoản phí nào. Đồng thời smart contract cũng không còn tiền trong tài khoản.