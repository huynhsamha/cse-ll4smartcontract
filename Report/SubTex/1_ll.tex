\subsection{Lịch sử và ứng dụng của Linear Logic}
\subsubsection{Lịch sử phát triển của Linear Logic}
Linear Logic (Logic tuyến tính) là một dạng Substructural Logic (Logic thiếu mất 1 trong những quy luật cấu trúc) được đề xuất bởi Jean-Yves Girald vào năm 1987 như là một sự cải tiến của Classical Logic (Logic cổ điển) và Institutional Logic (Logic mang tính trực giác), kết hợp giữa luật đối tính của Classical Logic với các quy luật hình thành của Institutional Logic.\\

Mục tiêu của Linear Logic là cầu kết nối giữa logic và khoa học máy tính vì nó cho phép thể hiện và điều khiển những sự kiện của thế giới thực một cách tự nhiên. Một ví dụ điển hình là Law of Excluded Middle (LEM). LEM nói rằng "có A hoặc không có A", vốn là một điều hoàn toàn hợp lý trong cuộc sống cùa chúng ta. Đối với Classical Logic, LEM là một luật được chấp nhận, tuy nhiên điều đó lại ngược lại đối với Institutional Logic. 

Linear Logic đã định nghĩa LEM theo 2 cách là $ A \oplus \neg A $ và $ A vẽ chữ and ngược (\& ngược) A $ .

Cách thứ nhất tương đương với công thức của phép tuyển trong Institutional Logic và cách thứ hai tương đương với "A suy ra A" là điều luôn đúng. Cả 2 cách này đều được chấp nhận trong Institutional Logic. Từ đó, ta có thể thấy Linear Logic là một sự kết hợp hoàn hảo giữa Classical và Institutional.


\subsubsection{Những ứng dụng của Linear Logic}
Linear Logic có rất nhiều công dụng trên nhiều lĩnh vực khác nhau. Ví dụ:


•	Điện toán lượng tử
Khác với máy tính thông thường sử dụng bit để lưu trữ các trạng thái, máy tính lượng tử sử dụng qubit. Ở máy tính thông thường, chi phí tính toán một phép toán một lần hay nhiều lần là như nhau. Vì vậy mà ta có thể sử dụng logic cổ điển để thể hiện các phép toán trên máy tính thông thường. Tuy nhiên, đối với máy tính lượng tử, việc thực hiện một phép toán nhiều lần sẽ có chi phí tính toán cao hơn là thực hiện chỉ 1 lần. Sử dụng Linear Logic sẽ thích hợp hơn trong trường hợp này vì việc quản lý tài nguyên là một vấn đề cần được xem xét.


•	Ngôn ngữ học
Linear Logic có thể được dùng để kiểm tra và chứng minh ngữ pháp và ngữ nghĩa của một ngôn ngữ. Như việc kiểm tra xem ngữ pháp của một ngôn ngữ có cấu trúc nhất định hay ngữ nghĩa của một câu nói có thể hiện được một điều đúng đắn trong logic.


•	Lập trình
Vì Linear Logic cho phép quản lý tài nguyên một cách hiệu quả hơn, chúng ta có thể áp dụng nó vào việc lập trình để quản lý bộ nhớ hay việc thu gom rác một cách hiệu quả hơn. Ngoài ra, Linear Logic giúp thể hiện những hiện tượng trong thế giới thực một cách tự nhiên hơn, giúp cho việc lập trình trở nên đa dạng và dễ dàng hơn.