\subsection{Mô tả ngữ cảnh cho smart contract}
\subsubsection{Động cơ và mục tiêu}
Trong các hoạt động kinh doanh, vận chuyển hàng hóa hiện nay, hợp đồng thông minh (smart contract) giúp giải quyết rất nhiều những khó khăn, rào cản mà việc xử lí thủ công mắc phải. Trước hết, bởi vì các smart contract sẽ thực hiện tự động những hoạt động mà trước đây thường phải thực hiện thủ công (như việc xác nhận đơn hàng) nên chúng có khả năng thúc đẩy tốc độ của quy trình kinh doanh. Ngoài ra, smart contract cũng đem lại sự đảm bảo về độ chính xác của giao dịch cao hơn, ít lỗi hơn, từ đó giảm thiểu đáng kể rủi ro khi thực hiện hợp đồng. Bên cạnh đó, một ưu điểm cần phải kể đến của ssmart contract đó là việc tối thiểu hóa sự tham gia của các bên thứ ba hay bên trung gian vào quá trình thực hiện hợp đồng. Sự tinh giản này giúp tiết kiệm đáng kể chi phí kinh doanh. Đặt trong bối cảnh một doanh nghiệp phải thực hiện hàng ngày một lượng khổng lồ các thủ tục xác nhận đơn hàng logistics, lợi ích mà smart contract đem lại sẽ không chỉ kể hết trong một hai trang giấy. Như vậy, với lý tưởng đảm bảo sự công bằng, minh bạch và tiết kiệm tối đa chi phí cho các bên tham gia giao kết hợp đồng, smart contract thực sự đã cho thấy những tiềm năng ứng dụng của nó, không chỉ đơn thuần áp dụng vào các hoạt động thương mại, mà còn được kỳ vọng sẽ hoạt động hiệu quả trong việc bầu cử, tiếp cận dữ liệu về sức khỏe và dân số, hỗ trợ quy trình bồi thường bảo hiểm hoặc quản lý chính phủ.

\subsubsection{Tình huống cụ thể}
Với tình huống cụ thể là giao dịch liên quan đến vận tải đa phương thức (logistics) giữa công ty A và công ty B. Giả sử bên A (công ty A) có nhu cầu vận chuyển một lô hàng thủ công mỹ nghệ, và sử dụng dịch vụ vận tải của công ty B (gọi là bên B). Tuy nhiên bên A và bên B chưa có được niềm tin tưởng lẫn nhau. Cụ thể bên A băn khoăn rằng liệu bên B có bảo quản tốt hàng hóa, đảm bảo về chất lượng lẫn số lượng, cũng như thời gian chuyển hàng tới địa điểm chỉ định đúng hẹn hay không? Còn bên B thì đặt ra câu hỏi là liệu bên A có đảm bảo uy tín, không phải là công ty lừa đảo, hay hàng hóa của bên A là hợp pháp hay bất hợp pháp, cũng như việc thù lao khi bên B giao hàng đến nơi có được chuyển đúng hẹn? Để giải quyết những khúc mắc này, 2 bên phải giao kết một hợp đồng thỏa thuận về việc vận chuyển hàng hóa, dẫn đến những phát sinh về dịch vụ tư vấn luật như soạn thảo hợp đồng, phòng ngừa các rủi ro pháp lý,... Vậy cách nào để đơn giản hóa quy trình thỏa thuận giữa 2 bên A và B, khi mà không có niềm tin trong môi trường doanh nghiệp như trong tình huống này?


Lúc này, smart contract có thể giải quyết câu hỏi này. Khi hai bên tham gia vào mô hình này, bên A sẽ thanh toán phí vận chuyển cho bên B thông qua smart contract. Bên B cũng đảm bảo số tiền đặt cọc bảo đảm về hàng hóa khi vận chuyển vào smart contract. Smart contract này sẽ giữ lại số tiền của cả 2 bên và việc thanh toán này chỉ được thực hiện khi 2 bên hoàn thành yêu cầu của hợp đồng kèm theo sự xác nhận của 2 bên. Chi tiết cụ thể như sau: 2 bên cung cấp những thông tin về thời gian, địa điểm, hàng hóa, các quy định bồi thường vào smart contract, và smart contract public thông tin này cho 2 phía. Ngoài ra 2 bên cùng chuyển vào smart contract số tiền cần thiết của 2 bên để smart contract giữ số tiền này. Cụ thể bên A cần chuyển vào số tiền thù lao mà bên A sẽ phải chi trả cho bên B nếu bên B hoàn tất công việc. Còn bên B cần chuyển vào số tiền mà bên B có thể sẽ phải bồi thường trong quá trình smart contract hoạt động, số tiền này tối thiểu phải là lượng tiền bồi thường lớn nhất mà bên B gánh phải trong suốt quá trình hợp đồng xảy ra. Khi 2 bên thỏa thuận, smart contract chính thức có hiệu lực. Trong suốt quá trình nhận hàng, kiểm tra hàng, vận chuyển, tiếp nhận, kiểm tra lại hàng hóa của 2 phía, smart contract sẽ luôn thực thi khi một điều kiện nào đó kích hoạt nó, bao gồm việc bồi thường của 2 bên, hoặc hợp đồng bị hủy, hoặc hoàn tất.


Chi tiết của các điều kiện trong bản hợp đồng sẽ kích hoạt smart contract như sau: diễn ra ở các giai đoạn B đến nhận hàng từ A, B kiểm tra hàng hóa của A, A nhận hàng khi B vận chuyển và cuối cùng là A kiểm tra lại hàng hóa của mình.

\begin{itemize}
	\item Khi B đến nhận hàng từ A\\
	Gồm các quy định về thời gian nhận hàng, B cần phải đến sớm hoặc đúng hơn thời gian nhận hàng, nếu đến trễ hoặc không đến, B sẽ phải bồi thường hoặc hủy hợp đồng. Bên A cũng có trách nhiệm có hàng hóa để giao cho bên B. Nếu bên A không có hàng hóa, bên A cũng chịu trách nhiệm bồi thường hoặc hủy hợp đồng.
	
	\item Khi B kiểm tra hàng hóa của A\\
	Gồm các quy định về thông tin hàng hóa, bên A phải đảm bảo hàng hóa của mình đúng như thông tin hàng hóa quy định trong smart contract. Bên A sẽ phải chịu bồi thường hoặc hủy hợp đồng nếu vi phạm
	
	\item Khi bên A nhận hàng hóa từ bên B\\
	Gồm các quy định về thời gian, bên B phải giao hàng cho bên A đúng địa điểm thời gian trong smart contract. Nếu đến trễ hoặc không đến, bên B sẽ phải chịu bồi thường.
	
	\item Khi bên A kiểm tra lại hàng hóa\\
	Gồm các quy định về hàng hóa, bên B phải đảm bảo hàng hóa còn nguyên vẹn, đảm bảo về chất lượng lẫn số lượng, vẫn còn niên phong nếu có. Nếu vi phạm, bên B sẽ phải chịu bồi thường trong smart contract. Nếu không có gì xảy ra, smart contract xem như hoàn tất và trả lại số tiền cho 2 bên đúng như quy định của các điều khoản trong hợp đồng.
\end{itemize}