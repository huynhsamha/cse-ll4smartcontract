\subsection{Lịch sử và ứng dụng của Smart Contract}
\subsubsection{Lịch sử phát triển và cái nhìn tổng quan về Smart Contract}
Theo dòng lịch sử chúng ta sẽ thấy công nghệ đã thay đổi rất nhiều thứ, kể cả nhận thức của mỗi chúng ta. Các công nghệ mới được sinh ra và dần dần thay thế các công nghệ đã trở nên lạc hậu. Bắt đầu kỉ nguyên của internet, niềm tin luôn là một thứ gì đó khá xa xỉ khi các tổ chức, cá nhân bắt tay, giao tiếp với nhau trên mạng internet. Đứng trước bài toán đó, vào những năm 1990, nhà mật mã học Nick Szabo đã đưa ra 1 khái niệm hoàn toàn mới là Smart Contract. Ông định nghĩa một smart contract là một giao thức máy tính tạo ra để số hóa, kiểm chứng và thực thi các thỏa thuận và nghĩa vụ được quy định trong một hợp đồng. Smart contract cho phép thực thi các giao dịch một cách đáng tin cậy mà không cần một bên thứ ba làm chứng. Các giao dịch đó có thể theo dấu và không thể đảo ngược được.\\

Một cách dễ hiểu để mô tả smart contract (dịch ra ngôn ngữ Việt là \textit{hợp đồng thôngn minh}) là hình dung công nghệ này như một máy bán nước tự động. Thông thường, người làm hợp đồng sẽ phải tìm đến luật sư hay đem đi công chứng, trả tiền cho họ và chờ đợi để lấy giấy tờ tài liệu. Bằng cách sử dụng hợp đồng thông minh, ta chỉ cần thả một bitcoin (giả sử pháp luật Việt Nam cho phép ta sử dụng đồng tiền kĩ thuật số Bitcoin trong giao dịch mua bán như trong ví dụ này) vào máy bán hàng tự động, đã được phát triển trên nền tảng Blockchain và những gì bạn yêu cầu sẽ được trả lại trực tiếp vào tài khoản của bạn. Hơn nữa, hợp đồng thông minh không chỉ xác minh những quy định, quyền lợi và nghĩa vụ giống như hợp đồng truyền thống mà nó còn tự động thực thi những điều trên, không thông qua một bên thứ 3 trong môi trường thiếu niềm tin.\\

Công nghệ nguyên thủy của hợp đồng thông minh đã từng là một bài toán tư duy \textit{"ngủ yên"} trong hơn một thập kỷ. Thế nhưng mọi thứ đã thay đổi với sự ra đời và phát triển của công nghệ Blockchain. Tuy Bitcoin đã đặt ra những nền tảng cơ bản cho việc thiết lập hợp đồng trên nền tảng Blockchain, nó vẫn còn chưa thể thõa mãn mọi nhu cầu trong đời sống xã hội hiện nay. Mãi cho đến khi Ethereum ra đời thì ý tưởng hợp đồng thông minh mới chính thức phổ biến, cung cấp phương thức mới đề thiết lập hợp đồng. Ngày nay, số lượng các tổ chức, công ty nghiên cứu về các lĩnh vực trong công nghệ Blockchain và ứng dụng của smart contract đã và đang tăng lên đáng kể, cho thấy sự phát triền đầy tiềm năng của công nghệ này.\\

\subsubsection{Những ứng dụng dựa trên Smart Contract}

Smart contract kết hợp với blockchain có thể ứng dụng trong rất nhiều lĩnh vực, từ tài chính, bảo hiểm, ngân hàng cho đến các nhu cầu như giải trí, bầu cử (voiting), bình chọn,... Với mỗi lĩnh vực ta lại sử dụng một loại hợp đồng thông minh khác nhau, ví dụ hợp đồng vay tiền, hợp đồng đóng tiền định kỳ, hợp đồng mua bán, hợp đồng chuyển nhượng,...\\

Khi một lĩnh vực được áp dụng hợp đồng thông minh hay còn gọi là smart contract, mọi quyết định hoặc giao dịch thuận theo hợp đồng đều được xử lý tự động khi thỏa điều kiện đã đưa ra. Ví dụ:

\begin{itemize}
	\item {Hợp đồng vay tiền có lãi suất:}\\
	Giả sử rằng bạn vay một số tiền có lãi suất và ký hợp đồng này, khi đến một ngày nhất định, hợp đồng thông minh sẽ tự trừ tiền trong tài khoản của bạn và gửi (cộng thêm) cho người cho vay theo tỉ lệ đúng như đã ký kết.
	
	\item{Hợp đồng chuyển nhượng:}\\
	Giả sử nếu bạn trả 50\% số tiền bạn sẽ được giữ cọc món hàng, nếu bạn trả đủ 100\% bạn sẽ được nhận món hàng đó, nếu bạn trả 50\% nhưng không thanh toán đủ trong thời gian quy định thì bạn sẽ mất cọc, tất cả đều được tự động xử lý với hợp đồng thông minh.
	
	\item{Bầu cử:}\\
	Giả sử ta đang xét trong một quá trình bầu cử cần sự minh bạch, rõ ràng và không xảy ra gian lận giữa người tổ chức và các ứng cử viên. Nếu ta thực hiện theo cách truyền thống, ví dụ như bỏ phiếu hoặc bình chọn bằng tin nhắn, có thể thấy còn thiếu sự minh bạch, và có thể xảy ra gian lận. Cụ thể người chơi (ứng cử viên) không thể biết được những ai đã bầu cho các người chơi khác, và những người bầu cử cũng không biết được thực sự những ai đã bầu cho những người nào. Do đó thiếu đi sự minh bạch trong phương thức truyền thống. Nếu ta sử dụng nền tảng công nghệ Blockchain với smart contract, mọi thứ sẽ được giải quyết ổn thỏa. Kết quả bỏ phiếu sẽ được chuyển vào Blockchain, do smart contract quản lý và phân phối về các node trong mạng lưới. Toàn bộ dữ liệu sẽ được mã hóa và hoàn toàn ẩn danh, mọi người cũng có thể biết chính xác có bao nhiêu phiếu đã bầu cử cho những người nào.
	
	\item{Logistics}\\
	Chúng ta đều biết rằng chuỗi cung ứng là một hệ thống kéo dài và gồm nhiều liên kết khác nhau. Mỗi liên kết cần phải nhận được xác nhận bởi một mắt xích (xem như một node) ở trước đó để đủ điều kiện thực hiện phần việc của mình theo như hợp đồng.
	
	Đây là một quá trình kéo dài, lãng phí và kém năng suất, nhưng với smart contract thì mỗi bộ phận tham gia đều có thể theo dõi tiến trình công việc để từ đó hoàn thành nhiệm vụ đúng hạn. Smart contract bảo đảm tính minh bạch trong điều khoản hợp đồng, chống gian lận.
	
	Nó còn có thể cung cấp cho ta khả năng giám sát quá trình cung ứng nếu như được tích hợp chung với mạng lưới vạn vật kết nối Internet (hay còn gọi là Internet of Things). Có thể nói hai công nghệ này kết hợp với nhau tạo ra một nền công nghệ 4.0 hoàn thiện.
	
	\item{ICO}\\
	Một ứng dụng rộng rãi của hợp đồng thông minh trong lĩnh vực tiền mã hóa đó là phát hành ICO. Một hợp đồng thông minh được viết ra để khi nhà đầu tư gửi vào địa chỉ của hợp đồng một số tiền, họ sẽ nhận lại được một số token tương ứng với số tiền họ đã bỏ ra. Hợp đồng thông minh đó có thể bổ sung một số điều kiện như đóng băng số tiền nhận được trong một khoản thời gian quy định hoặc hủy các token không bán được.
	
	Việc ứng dụng hợp đồng thông minh và blockchain trong ICO giúp các nhà đầu tư có thể theo dõi dự án đã gọi được bao nhiêu tiền, có đạt được mục tiêu hay không, và nhiều thông tin khác nữa.
	
\end{itemize}

Ngoài ra, smart contract còn có rất nhiều những ứng dụng khác trong đời sống xã hội cũng như kinh tế chính trị, lĩnh vực tài chính,...